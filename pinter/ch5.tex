\documentclass[12pt]{article}
\usepackage{amssymb,amsmath,latexsym}

\begin{document}
\title{A Book of Abstract Algebra: Solutions to Chapter 5}
\author{Tushar Tyagi}
\date{\today}
\maketitle

\section*{Notes}
A subgroup $S$ is called a subgroup of a group $G$, if:
\begin{enumerate}
  
\item It is closed on the given operation, i.e. the operation ($\cdot$) of two elements produces an element $\in$ $S$.
\item It is closed under inverse, i.e. the inverse of each element of $S$ is in $S$.
\end{enumerate}

Also, each subgroup is a group as well, and therefore follows the three group laws:
\begin{enumerate}
  \item Associativity
  \item Identity
  \item Inverse
\end{enumerate}

The $identity, e$ of the group is shared by the subgroup.

\subsubsection*{Trivial \& Proper Subgroups}
\begin{enumerate}
 \item The one-element subset $\{e\}$ and the entire group $G$ are the smallest and the largest subgroups of $G$ and are called \textit{trivial subgroups}.
 \item All the other subgroups of G are called \textit{proper subgroups}.
\end{enumerate}


\subsubsection*{Cyclic Groups and Subgroups}
If a group (or a subgroup) is generated by a single element, we call that group
\textit{Cyclic} and it is written as $\langle a \rangle $, where $a$ is called the \textit{generator} and is the single element which, along with the identity and $a^{-1}$, can define the entire group. 


\subsubsection*{Defining Equations}
A set of equations, involving only the generators and their inverses, is called a set of \textit{defining equations}. These equations can completely define the operation table of the group.


\section*{Solutions}
\label{sec:solutions}

\subsection*{Set A}

\begin{enumerate}
\item $G = \langle R, + \rangle, H = \{log a: a \in \mathbb{Q}, a > 0\}$
  \begin{itemize}
    \item 
      Addition:
      
      Let $a, b \in \mathbb{Q} \\
      log\ a + log\ b = log\ ab \\
      \because a, b \in \mathbb{Q},\\
      \therefore ab \in \mathbb{Q}, ab > 0, \\
      \Rightarrow log\ ab \in H$
    \item 
      Identity:
      
      The identity element would not change the value of $log\ a$ under addition. 
      $log\ 1$ or $0$ is the identity element, since:

      If $log\ a + log\ b = log\ a$, then $log\ b = 0$, and $b = 1$.


    \item 
      Inverse:
        \begin{alignat*}{3}
          &log\ a + log\ a^{-1} \ &= &\ e \\ 
          \Rightarrow & log\ a &= & -log\ a^{-1} \\
          \Rightarrow & log a   &= &\ log(\frac{1}{a^{-1}}) \\
          \Rightarrow & a      &= &\ \frac{1}{a^{-1}}
        \end{alignat*}
      
      Since $a \in \mathbb{Q}$, $\frac{1}{a^{-1}} \in \mathbb{Q}$, $\therefore log\ a^{-1} \in H$
       
    \end{itemize}

  \end{enumerate}


\end{document}
